% Created with jtex v.1.0.20
\documentclass[11pt]{article}
\usepackage{
    amsmath,
    fancyhdr,
    geometry,
    lastpage,
    xcolor
}

% You should have an imports section
%%%%%%%%%%%%%%%%%%%%%%%%%%%%%%%%%%%%%%%%%%%%%%%%%%
%%%%%%%%%%%%%%%%%%%%  imports  %%%%%%%%%%%%%%%%%%%
\usepackage{booktabs}
\usepackage{framed}
\usepackage{hyperref}
\usepackage{url}
%%%%%%%%%%%%%%%%%%%%%%%%%%%%%%%%%%%%%%%%%%%%%%%%%%


% Set document format
\setlength{\parindent}{0pt} % remove indent for paragraph
\setlength{\parskip}{6pt}   % change paragraph skip
\setcounter{secnumdepth}{0} % remove section numbering
\makeatletter
\renewcommand\section{\@startsection{section}{1}{0pt}%
  {-3.5ex \@plus -1ex \@minus -.2ex}% Space before
  {.2ex \@plus.2ex}% Space after
  {\normalfont\Large\bfseries}} % Style of the section
\makeatother

% Page setup
 \geometry{
    letterpaper,
    left=0.5in,
    right=0.5in,
    top=0.75in,
    bottom=1in
 }

\pagestyle{fancy}
\fancyhead{} % clear header
\fancyfoot{} % clear footer
\fancyhead[R]{MAT 171-OSI2 (2025SP) Course Syllabus}
\fancyfoot[R]{Page \thepage\,of\,\pageref{LastPage}}
\fancyfoot[L]{Date Generated: \today}

% Define colors
\definecolor{mybgcolor}{RGB}{240, 240, 255} % Light pastel blue
\definecolor{mybordercolor}{RGB}{0, 51, 102} % Navy blue

\setlength{\FrameSep}{8pt} % Padding inside the frame
\setlength{\OuterFrameSep}{6pt} % Margin outside the frame
\renewenvironment{framed}[1][]{%
  \def\FrameCommand{%
    \hspace{1pt}%
    {\color{mybordercolor}\vrule width 2pt} % Border color and thickness
    \hspace{1pt}%
    \fboxsep=\FrameSep%
    \colorbox{mybgcolor}%
  }%
  \MakeFramed {\advance\hsize-\width \FrameRestore}%
}{%
  \endMakeFramed
}

% Define document parts/options
\title{MAT 171-OSI2 (2025SP)}
% \abstract{}

\begin{document}

\begin{center}
    {\huge Mitchell Community College} \\[6pt]
    {\Large Precalculus Algebra} \\[6pt]
    {\Large MAT 171-OSI2 (2025SP)}
\end{center}

\section{Course Information}

\textbf{Course Delivery:} This course is delivered online with asynchronous lectures. There are two required, in-person dates for the Midterm and Final Exam.

\textbf{Class Location(s)/Day(s)/Time(s):} Lectures are provided online as prerecorded lectures. There are no scheduled meeting times for class/lecture times.

\begin{itemize}
\item An in-person Midterm Exam will be given on Friday, March 7, 2025 from 9:30 a.m. until 10:45 a.m. on the Statesville Campus of Mitchell Community College in Bently Building, room 209\footnote{The Bently Building was previously known as the Vocational Building (VB).}.
\item An in-person Final Exam will be given on Friday, May 9, 2025 from 9:30 a.m. until 10:45 a.m. on the Statesville Campus of Mitchell Community College in Bently Buildin, room 209.
\end{itemize}

\textbf{Institutional Credit Hours/Contact Hours:} 4 Credit, 3 Class, 2 Lab

\textbf{Prerequisites:} Please refer to the Mitchell Community College Catalog for more information.

\textbf{No Show Date (Census Date}): 1/26/2025

\section{Faculty Information}

\textbf{Instructor:} Ethan A. Smith, MA, MEd

\textbf{Email Address:} \href{mailto:esmith3@mitchellcc.edu}{esmith3@mitchellcc.edu}

\textbf{Response Time:} Mr. Smith will reply to emails and phone calls within 36 hours Monday through Friday. During the weekends, holidays, and institutional closing, response times may be increased due to lack of access to technology.

\textbf{Phone Number:} (704) 878-3270

\textbf{Office Location:} Bently Building, room 212A

\textbf{Available for Student Support:}

\begin{itemize}
\item Monday and Wednesday: 12 p.m. -- 1 p.m.
\item Tuesday and Thursday: 12:30 p.m. -- 2 p.m.
\end{itemize}

\textbf{College Reception Desks:}

\begin{itemize}
\item Statesville Campus: (704) 878-3200
\item Mooresville Campus: (704) 663-1923
\end{itemize}

\section{Catalog Description}

This course is designed to develop topics which are fundamental to the study of Calculus. Emphasis is placed on solving equations and inequalities, solving systems of equations and inequalities, and analysis of functions (absolute value, radical, polynomial, rational, exponential, and logarithmic) in multiple representations. Upon completion, students should be able to select and use appropriate models and techniques for finding solutions to algebra-related problems with and without technology. This course has been approved for transfer under the CAA as a general education course in Mathematics. This course has been approved for transfer under the ICAA as a general education course in Mathematics.

\section{Learning Management System (LMS)}

All curriculum courses have a space on Mitchell Community College's learning management system (LMS). The College's LMS is Moodle, which is hosted by Open LMS. Moodle is the college-wide adopted delivery portal for online course content. The College's Moodle site can be accessed from the \href{https://mitchellcc.edu/my-mitchell}{My Mitchell} page on the Mitchell Community College website or at \href{https://mycourses.mitchellcc.edu}{https://mycourses.mitchellcc.edu} by direct link.

\section{Learning Outcomes}

\begin{itemize}
\item \textbf{SLO1:} Use analytical, graphical, and numerical representations to solve absolute value, radical, polynomial, rational, exponential, and logarithmic equations with both real and complex solutions.
\item \textbf{SLO2:} Use analytical, graphical, and numerical representations to solve absolute value, polynomial and rational inequalities with real solutions.
\item \textbf{SLO3:} Use analytical, graphical, and numerical representations to analyze absolute value, radical, polynomial, rational, exponential and logarithmic functions with both real and complex zeros.
\item \textbf{SLO4:} Use multiple methods to solve problems involving systems of equations.
\item \textbf{SLO5:} Apply multiple methods to decomposing partial fractions.
\item \textbf{SLO6:} Construct the composition and inverse of functions.
\item \textbf{SLO7:} Use polynomial, exponential and logarithmic functions to model various real-world situations in order to analyze, draw conclusions, and make predictions.
\item \textbf{SLO8:} Apply quantitative reasoning to solve problems.
\end{itemize}

\section{Mitchell Essential Learning Outcome (MELO)}

This course may assess the Mitchell Essential Learning Outcome (MELO) of Quantitative Inquiry.

\section{Recommendations for Success}

The definition of success can mean something different depending on the course. If you show up and engage with the material from the course, I will work with you to guide you to the goals and kinds of success you work to achieve. It is important to remember the time~requirements~for~a~course for a course when making a plan for success.

Some prior knowledge of basic mathematics and number sense can help to increase your success.

Below are some other recommendations for success in an online course from \href{https://online.osu.edu/resources/learn/5-online-learning-tips-student-success}{The Ohio State University} and the \href{https://www.colorado.edu/health/tips-succeeding-online-classes}{University of Colorado Boulder} .

\begin{enumerate}
\item Treat your online courses the same as in-person classes.
\item Familiarize yourself with the technology.
\item Hold yourself accountable
\item Create a schedule and manage your time wisely.
\item Stay organized and be thorough.
\item Remain engaged throughout the whole course.
\item Take care of yourself.
\item Know where to turn for help.
\end{enumerate}

\begin{framed}
\textbf{Time Requirements of a College-level Course}\\
The amount of time a student is expected to dedicate to studying and completing assignments for a specific course, typically includes class time, homework, reading, and preparation for exams. As guideline, for traditional, 16-week courses, one credit hour equates to three hours per week completing course activities. Therefore, if you are taking a 16-week course worth four credit hours, you should expect to spend 12 hours per week working in the course. This time allotment increases as the number of weeks of the course decreases. Below is a table showing examples of various hours per week that is recommended to devote to the course based on the length of the course. You should be aware that these are \textit{suggested} times based on the credit hours and the length of the course. Various factors such as previous knowledge, difficulty of material, and extracurricular activites may alter these times. There is not a one-size-fit-all approach or rule for planning your success in a college-level course.

\bigskip\noindent
\begin{tabular}{p{\dimexpr 0.200\linewidth-2\tabcolsep}p{\dimexpr 0.200\linewidth-2\tabcolsep}p{\dimexpr 0.200\linewidth-2\tabcolsep}p{\dimexpr 0.200\linewidth-2\tabcolsep}p{\dimexpr 0.200\linewidth-2\tabcolsep}}
\toprule
Credit Hours & 4-week Course & 8-week Course & 12-week Course & 16-week Course \\
\hline
1 & 12 & 6 & 4.5 & 3 \\
2 & 24 & 12 & 9 & 6 \\
3 & 36 & 18 & 13.5 & 9 \\
4 & 48 & 24 & 18 & 12 \\
5 & 60 & 30 & 22.5 & 15 \\
\bottomrule
\end{tabular}

\bigskip
\end{framed}

\section{Required Textbook and Other Materials}

\subsection{Texbook}

Sullivan, M., M. Sullivan III. (2021). \textit{Precalculus Enhanced with Graphing Utilities (8th)}. Boston: Pearson.

Note that a hard copy version of the textbook is optional, since it is available in electronic format with purchase of a MyMathLab Student Access Code.

\subsection{Optional (Free) Resources}

\begin{itemize}
\item Abramson J. (2021). \textit{Algebra and trigonometry 2e}. OpenStax. (Freely available to read and download at \href{https://openstax.org/details/books/algebra-and-trigonometry-2e.}{openstax.org})


\item \href{https://youtube.com/playlist?list=PLDesaqWTN6ESsmwELdrzhcGiRhk5DjwLP\&si=G7hNZvDueWkJ9Ydf}{Professor Leonard's Precalculus Lecture Series on YouTube}
\end{itemize}

\subsection{Calculator}

\textbf{Physical Graphing Calculator.} TI-83/84 Plus family (or less) is recommended. Students may not use a calculator with a computer algebra system (CAS) built into the calculator.

\textbf{Desmos Graphing Calculator.} Students in this section of MAT-171 are allowed to use the Desmos Graphing Calculator during their assignments and during Calculator Active portions of their Midterm Exam and Final Exam. Students can access the Desmos Graphing Calculator at \href{https://www.desmos.com/calculator}{desmos.com/calculator}.

\subsection{MyMathLab}

You must purchase access to MyMathLab. See the Assignment Description of this syllabus for more information about MyMathLab.

\section{Scanner or Scanner Software}

You will need to have access to a scanner or a scanner application on your mobile device in order to upload certain assignments for this course. There are several free scanner apps inside of the various ``stores'' for both Android and Apple devices. If you need assistance with scanning assignments onto your phone and uploading it to the respective place inside of OpenLMS, you will need to reach out to your instructor for assistance.

\section{Teaching and Learning Strategies}

Mitchell uses Contextual Teaching and Learning (CTL) to enhance student learning through activities that connect academic concepts to relevant life experiences.

\section{Learning Environment}

The lectures, worksheets, and homework assignments are accessed through the online learning environment, OpenLMS. Except for the Midterm Exam and Final Exam, this course does not take place in real--time. Students are provided with content such as videos, worksheets, and tests that are given a time frame to complete. Interaction with each other and the instructor, in most cases, takes place through OpenLMS and email. As a result, there is no class meeting time (with the exception of the Midterm Exam and the Final Exam). Asynchronous online learning environments are effective for students with time constraints or busy schedules. However, is imperative that students in an online course follow the Recommendations~for~Success and seek out help and assistance when needed.

\section{Technology Requirements and Resources}

For information on Technology Requirements, visit the Mitchell Community College \href{https://www.mitchellcc.edu/college-credit-online-classes/\#online-learning}{Technology Requirements and Expectations} webpage (opens in a new window).

\section{Grading}

\subsection{Grading Scale}

\bigskip\noindent
\begin{tabular}{p{\dimexpr 0.500\linewidth-2\tabcolsep}p{\dimexpr 0.500\linewidth-2\tabcolsep}}
\toprule
Numeric Grade & Letter Grade \\
\hline
90-100\% & A \\
80-89\% & B \\
70-79\% & C \\
60-69\% & D \\
0-59\% & F \\
\bottomrule
\end{tabular}

\bigskip\subsection{Grading Breakdown}

Your grade is based on the following weighted categories and corresponding percentages.

\begin{itemize}
\item Homework (MyMathLab) (15\%)
\item Worksheets (15\%)
\item Unit Tests (30\%)
\item Midterm Exam (20\%)
\item Final Exam (20\%)
\end{itemize}

\subsection{Assignment Descriptions}

A brief description of assignments is provided below. Detailed information and requirements will be provided in class and/or online within the Learning Management System (LMS). Each of the following assignments are expected to be submitted to their respective location within the LMS. No submission through email will be accepted.

\textbf{Homework.} Homework assignments will be given through MyMathLab. MyMathLab is an online learning platform that combines trusted author content with digital tools specifically designed to supplement mathematics textbooks and instruction. The questions in MyMathLab are randomized for each student; therefore, it is highly unlikely that any two students will have the same exact MyMathLab problems. MyMathLab assignments must be completed by the due date listed in MyMathLab (which should be no later than the date of the unit test) in order to receive full credit. Students are able to receive 75 percent credit on any MyMathLab problems completed after the due date. The final submission date to receive partial credit for MyMathLab is the Last Day of Classes.

\textbf{Worksheets.} Students should expect worksheets questions to be any combination of multiple choice and/or free response. Worksheet assignments should be individual work. All written assignments should be neat and legible. Note that any calculator with a computer algebra system such as a TI-89 or TI-Inspire CAS will not be allowed on worksheets. Worksheets must be scanned and uploaded to Open LMS to the correct assignment location to receive credit.

\begin{framed}
\textbf{Submitting Worksheets for Grading}\\
Only submissions submitted through the link in the LMS will be graded. Submissions sent via email will not be accepted or graded. If you experience issues submitting your worksheet assignments, please email your instructor as soon as possible. See Faculty~Information for your instructor's contact information.
\end{framed}

\textbf{Unit Tests.} Tests will be administered through MyMathLab. Tests will include multiple choice and/or free response questions. Each unit test has a time limit. The time limit for each test is posted in the description of the test and displayed when you click to attempt the test. Each test has a password. By entering the password of the unit test you are agreeing that you will abide by the Academic Integrity Policy as outline in the course syllabus and the Mitchell Community College Student Handbook. The password also acts as a safeguard against accidentally starting your test attempt. Any calculator with a computer algebra system such as a TI-89 or TI-Inspire are not be allowed on tests.

\textbf{Midterm.} Students are expected to come to campus to complete the Midterm Exam (see the course's description in Self-Service for finalized Midterm Exam dates). The Midterm Exam will cover the first half of the course (Units 1, 2, and 3). The Midterm Exam will contain two parts: a calculator inactive portion and a calculator active portion.

\textbf{Final Exam.} Students are expected to come to campus to complete the Final Exam (see the course's description in Self-Service for finalized Final Exam date). The Final Exam will contain materials from the second half of the course (Units 4, 5, and 6). The Final Exam will contain two parts: a calculator inactive portion and a calculator active portion.

\subsection{Missed Work}

Any assignments not completed by the due date is considered late. See Late~Work policy.

\subsection{Late Work}

There are no make-ups for missed worksheets, quizzes, the Midterm Exam, or the Final Exam unless arrangements are made with the instructor prior to the due date. Prior arrangements, except for extreme, extenuating circumstance, is considered to be 24 hours before the due date of the assignment for which an extension is being requested. In most cases, request for extensions after a due date will not be granted.

For Unit Tests, students who request an extenstion within 48 hours (this includes 48 hours before and 48 hours after) the due date will be granted an extension not exceeding 48 hours.

\section{Faculty Feedback and Response Time}

\subsection{Grading and Feedback}

For most assessments, you can generally expect feedback within seven days. The only exception to this feedback/response time would be when an institutional holiday occurs.

Feedback on your MyMathLab will be automatically generated by the software. If you are unclear why you missed a problem, you should reach out to your instructor.

Feedback on all handwritten assignments such as in-class worksheets, quizzes, and tests will be handwritten and returned either in class or electronically via the Learning Management System.

\section{Attendance Policy}

Mitchell Community College is an attendance taking institution. Instructors in all curriculum courses are required to report student attendance. Attendance begins on the first scheduled day of a course, even for students who register late. Mitchell Community College recognizes the connection between student attendance and student retention, achievement, and success. Students are expected to attend all class sessions, clinical experiences, and laboratory periods for which they are enrolled. Absence from any of these learning experiences, regardless of cause, reduces the opportunity for learning and may adversely affect a student's achievement.

Students are responsible for class attendance and for any class work missed during an absence. When a student fails to comply with the attendance policy of the class or fails to attend for two consecutive weeks (14 consecutive calendar days), the instructor should process an administrative withdrawal for the student resulting in a grade of W.

\subsection{Mathematics Department Attendance Policy for Online Courses}

If a student fails to be active in the course for 14 consecutive calendar days, then they will be withdrawn from the course.

\begin{framed}
\textbf{Defining Active Participation}\\
A student is considered active in this online course if they submit an assignment or make progress towards successfully completing assessments of learning.

Activities that \textbf{do} count towards active participation include

\begin{itemize}
\item Making progress in a MyMathLab assignment for that week.
\item Submitting a worksheet by the posted due date.
\item Completing an attempt on a Unit Test.
\item Sitting for an in-person Midterm or Final Exam.
\end{itemize}

Activities that \textbf{do not} count towards active participation include

\begin{itemize}
\item Only logging into the LMS without doing any of the above activities.
\item Watching a lecture video.
\item Clicking on an assignment, but not submitting any work.
\end{itemize}
\end{framed}

\section{Mitchell Community College Inclement Weather Policy}

In the event of adverse weather, the College will announce delays, cancellation of classes, or the closing of the college on local television and radio stations and on the \href{https://www.mitchellcc.edu}{College website}.

\section{No Show Date/Census Date Policy}

In order to remain enrolled in a course, a student must attend class on or before the class census date. If a student does not attend class by the census date, they will be reported as a \textbf{``no show'' (NS)} by the instructor and will be automatically withdrawn from the course.  To ensure students attend class and avoid being marked as a ``no show'', students need to:

\begin{itemize}
\item For traditional 100\% seated classes, a student must be physically present in class on or before the class census date.


\item For blended or hybrid classes, a student must either complete the Enrollment Verification Activity (EVA) in the LMS or physically be present in class on or before the class census date.


\item For 100\% online classes, a student must complete the EVA in the LMS on or before the class census date.
\end{itemize}

If a student does not meet the census date requirement, the student must be reported as a no-show for the class. Students reported as a ``no-show'' are withdrawn from the class.  The no show date and the census date are the same date for a course and can be found on the course syllabus as well as on the \href{https://mitchellcc.edu/office-student-records}{Office of Student Records webpage} (link opens in a new window). For blended, hybrid, and online courses this date is also noted in the EVA.

\section{Withdrawal Policy}

The last day a student can withdraw from a course or from all courses with a grade of ``W'' is at the 75 percent point of the course. The exact date is published on the \href{https://www.mitchellcc.edu/wp-content/uploads/2024/05/2024-2025-Academic-Calendar.pdf}{Academic Calendar} (link opens in new window). After the 75 percent point of the course, the student can no longer initiate a withdrawal and will receive the grade earned in the course at the end of the term. Students can locate withdrawal information by visiting the Office of Student Records webpage or the \href{https://www.mitchellcc.edu/schedule-adjustments-and-class-withdrawals/}{Schedule Adjustments and Class Withdrawals page}. If they need to withdraw from \textbf{all} of their classes they can click on the ``\href{https://mitchellcc.edu/withdrawal-procedure}{Withdrawal Procedure}'' menu link (links opens in a new window).

\section{Academic Dishonesty}

Mitchell Community College makes every reasonable effort to maintain integrity in all academic programs. To compromise integrity through acts of academic dishonesty jeopardizes the quality of instruction and the caliber of education we aim to provide our students.  Any form of academic dishonesty, by any student at the College, is unacceptable and will result in disciplinary action.

Definitions and Examples of Academic Dishonesty include, but may not be limited to:

\textbf{Cheating.} Intentionally and/or knowingly using unauthorized materials, information, or study aids in any academic exercise or matter.

\textbf{Plagiarism.} Intentionally and/or knowingly representing the words or ideas of another as one's own in any academic exercise or matter.

\textbf{Fabrication.} Intentionally and/or knowingly falsifying or inventing information or citations in an academic exercise or matter.

\textbf{Facilitating Academic Dishonesty.} Intentionally and/or knowingly helping or attempting to help another to commit an act of cheating, plagiarism, or fabrication.

\textbf{Self-Plagiarism.} The use of one's own previous work in another context without citing that it was used previously. The writer should let the reader know that this was not the first use of the material.

\pagebreak
\section{Tentative Schedule}

The following is a tentative schdule of the assigned readings and/or assessments for the course. All of these dates and learning activities are subject to change. The instructor will notify students of any change through email and the Announcements section of the Learning Management System (LMS) class site.

\textbf{Monday, January 13}

\begin{itemize}
\item First Day of Class
\item Read the Course Syllabus
\item Take the Enrollment Verification Assignment (EVA), previously known as the Mandatory Course Enrollment Activity
\end{itemize}

\textbf{Tuesday, January 14}

\begin{itemize}
\item Work on Section 2.1
\end{itemize}

\textbf{Wednesday, January 15}

\begin{itemize}
\item Complete Section 2.1
\item Complete Worksheet \#1: Scanning Activity
\end{itemize}

\textbf{Thursday, January 16}

\begin{itemize}
\item Work on Section 2.2
\end{itemize}

\textbf{Friday, January 17}

\begin{itemize}
\item Complete on Section 2.2
\end{itemize}

\textbf{Monday, January 20}

\begin{itemize}
\item \textbf{Martin Luther King Jr. Day | College Closed | No Class}
\end{itemize}

\textbf{Tuesday, January 21}

\begin{itemize}
\item Work on Section 2.3
\item \textbf{Submission Deadline for Worksheet \#1: Scanning Activity}
\end{itemize}

\textbf{Wednesday, January 22}

\begin{itemize}
\item Complete Section 2.3
\end{itemize}

\textbf{Thursday, January 23}

\begin{itemize}
\item Complete Section 2.4
\end{itemize}

\textbf{Friday, January 24}

\begin{itemize}
\item Work on Section 2.5
\item Complete Worksheet \#2
\end{itemize}

\textbf{Monday, January 27}

\begin{itemize}
\item Review for Unit 1 Test
\end{itemize}

\textbf{Tuesday, January 28}

\begin{itemize}
\item Take Unit 1 Test or Catch-up
\end{itemize}

\textbf{Wednesday, January 29}

\begin{itemize}
\item Work on Section 1.5
\end{itemize}

\textbf{Thursday, January 30}

\begin{itemize}
\item Complete Section 1.5
\end{itemize}

\textbf{Friday, January 31}

\begin{itemize}
\item Complete Section 3.1
\end{itemize}

\textbf{Monday, February 3}

\begin{itemize}
\item \textbf{Submission Deadline for Worksheet \#2}
\item \textbf{Submission Deadline for Unit 1 Test}
\item \textbf{Submission Deadline for Unit 1 MyMathLab for full credit. Attempts for Unit 1 MyMathLab submitted after this date will recieve a 25\% point deduction.}
\end{itemize}

\textbf{Tuesday, February 4}

\begin{itemize}
\item Work on Section 3.2
\end{itemize}

\textbf{Wednesday, February 5}

\begin{itemize}
\item Complete Section 3.2
\end{itemize}

\textbf{Thursday, February 6}

\begin{itemize}
\item Work on Section 11.1
\end{itemize}

\textbf{Friday, February 7}

\begin{itemize}
\item Complete Section 11.1
\end{itemize}

\textbf{Monday, February 10}

\begin{itemize}
\item Work on Section 11.2
\end{itemize}

\textbf{Tuesday, February 11}

\begin{itemize}
\item Complete Section 11.2
\end{itemize}

\textbf{Wednesday, February 12}

\begin{itemize}
\item Complete Worksheet \#3
\end{itemize}

\textbf{Thursday, February 13}

\begin{itemize}
\item Catch-up Day
\end{itemize}

\textbf{Friday, February 14}

\begin{itemize}
\item Review for Unit 2 Test
\item Review for Unit 2 Test or Take Unit 2 Test
\end{itemize}

\textbf{Monday, February 17}

\begin{itemize}
\item \textbf{Submission Deadline for Unit 2 Test}
\item \textbf{Submission Deadline for Unit 2 MyMathLab for full credit. Attempts for Unit 2 MyMathLab submitted after this date will recieve a 25\% point deduction.}
\item \textbf{Submission Deadline for Worksheet \#3}
\end{itemize}

\textbf{Tuesday, February 18}

\begin{itemize}
\item Work on Section 3.3
\end{itemize}

\textbf{Wednesday, February 19}

\begin{itemize}
\item Complete Section 3.3
\end{itemize}

\textbf{Thursday, February 20}

\begin{itemize}
\item Work on Section 3.4
\end{itemize}

\textbf{Friday, February 21}

\begin{itemize}
\item Work on Section 3.4
\end{itemize}

\textbf{Monday, February 24}

\begin{itemize}
\item Complete Section 3.4
\end{itemize}

\textbf{Tuesday, February 25}

\begin{itemize}
\item Work on Section 3.5
\end{itemize}

\textbf{Wednesday, February 26}

\begin{itemize}
\item Complete Section 3.5
\end{itemize}

\textbf{Thursday, February 27}

\begin{itemize}
\item Complete Worksheet \#4
\end{itemize}

\textbf{Friday, February 28}

\begin{itemize}
\item Review for Unit 3 Test
\item Take Unit 3 Test
\end{itemize}

\textbf{Monday, March 3}

\begin{itemize}
\item \textbf{Submission Deadline for Worksheet \#4}
\item \textbf{Submission Deadline for Unit 3 Test}
\item \textbf{Submission Deadline for Unit 3 MyMathLab for full credit.} Attempts for Unit 3 MyMathLab submitted after this date will recieve a 25\% point deduction.
\end{itemize}

\textbf{Tuesday, March 4}

\begin{itemize}
\item Review for the in-person Midterm Exam
\item Work missed MyMathLab (penalty will apply)
\end{itemize}

\textbf{Wednesday, March 5}

\begin{itemize}
\item Review for the in-person Midterm Exam
\item Work missed MyMathLab (penalty will apply)
\end{itemize}

\textbf{Thursday, March 6}

\begin{itemize}
\item Review for the in-person Midterm Exam
\item Work missed MyMathLab (penalty will apply)
\end{itemize}

\textbf{Friday, March 7}

\begin{itemize}
\item In-person Midterm Exam
\end{itemize}

\textbf{Monday, March 10}

\begin{itemize}
\item \textbf{Spring Break | No Class | College Open}
\end{itemize}

\textbf{Tuesday, March 11}

\begin{itemize}
\item \textbf{Spring Break | No Class | College Open}
\end{itemize}

\textbf{Wednesday, March 12}

\begin{itemize}
\item \textbf{Spring Break | No Class | College Open}
\end{itemize}

\textbf{Thursday, March 13}

\begin{itemize}
\item \textbf{Spring Break | No Class | College Open}
\end{itemize}

\textbf{Friday, March 14}

\begin{itemize}
\item \textbf{Spring Break | No Class | College Open}
\end{itemize}

\textbf{Monday, March 17}

\begin{itemize}
\item Complete on Section 4.1
\end{itemize}

\textbf{Tuesday, March 18}

\begin{itemize}
\item Complete Section 4.2
\end{itemize}

\textbf{Wednesday, March 19}

\begin{itemize}
\item Work on Section 4.3
\end{itemize}

\textbf{Thursday, March 20}

\begin{itemize}
\item Complete Section 4.3
\end{itemize}

\textbf{Friday, March 21}

\begin{itemize}
\item Work on Section 4.4
\item Complete Worksheet \#5
\end{itemize}

\textbf{Monday, March 24}

\begin{itemize}
\item Complete Section 4.4
\end{itemize}

\textbf{Tuesday, March 25}

\begin{itemize}
\item Work on Section 4.7A
\end{itemize}

\textbf{Wednesday, March 26}

\begin{itemize}
\item Complete Section 4.7A
\end{itemize}

\textbf{Thursday, March 27}

\begin{itemize}
\item Catch-up Day
\end{itemize}

\textbf{Friday, March 28}

\begin{itemize}
\item Review for Unit 4 Test
\item Take Unit 4 Test
\end{itemize}

\textbf{Monday, March 31}

\begin{itemize}
\item \textbf{Submission Deadline for Worksheet \#5}
\item \textbf{Submission Deadline for Unit 4 Test}
\item \textbf{Submission Deadline for Unit 4 MyMathLab for full credit. Attempts for Unit 4 MyMathLab submitted after this date will recieve a 25\% point deduction.}
\end{itemize}

\textbf{Tuesday, April 1}

\begin{itemize}
\item Work on Section 4.5
\end{itemize}

\textbf{Wednesday, April 2}

\begin{itemize}
\item Complete Section 4.5
\end{itemize}

\textbf{Thursday, April 3}

\begin{itemize}
\item Work on Section 4.6
\end{itemize}

\textbf{Friday, April 4}

\begin{itemize}
\item Work on Section 4.7B
\end{itemize}

\textbf{Monday, April 7}

\begin{itemize}
\item Work on Section 4.7B
\item Complete Worksheet \#6
\end{itemize}

\textbf{Tuesday, April 8}

\begin{itemize}
\item Complete Section 11.5
\end{itemize}

\textbf{Wednesday, April 9}

\begin{itemize}
\item Review for Unit 5 Test
\item Take Unit 5 Test
\end{itemize}

\textbf{Thursday, April 10}

\begin{itemize}
\item Complete Section 5.1
\end{itemize}

\textbf{Friday, April 11}

\begin{itemize}
\item Work on Section 5.2
\end{itemize}

\textbf{Monday, April 14}

\begin{itemize}
\item Complete Section 5.2
\item Work on Section 5.3
\item Work on Worksheet \#7
\item \textbf{Submission Deadline for Worksheet \#6}
\item \textbf{Submission Deadline for Unit 5 Test}
\item \textbf{Submission Deadline for Unit 5 MyMathLab for full credit. Attempts for Unit 5 MyMathLab submitted after this date will recieve a 25\% point deduction.}
\end{itemize}

\textbf{Tuesday, April 15}

\begin{itemize}
\item Complete Section 5.3
\end{itemize}

\textbf{Wednesday, April 16}

\begin{itemize}
\item Work on Section 5.4
\end{itemize}

\textbf{Thursday, April 17}

\begin{itemize}
\item Complete Section 5.4
\end{itemize}

\textbf{Friday, April 18}

\begin{itemize}
\item Work on Section 5.5
\end{itemize}

\textbf{Monday, April 21}

\begin{itemize}
\item \textbf{Submission Deadline for Worksheet \#7}
\item Complete Section 5.5
\end{itemize}

\textbf{Tuesday, April 22}

\begin{itemize}
\item Work on Section 5.6
\end{itemize}

\textbf{Wednesday, April 23}

\begin{itemize}
\item Complete Section 5.6
\item Complete Worksheet \#8
\end{itemize}

\textbf{Thursday, April 24}

\begin{itemize}
\item Work on Section 5.7
\end{itemize}

\textbf{Friday, April 25}

\begin{itemize}
\item Complete Section 5.7
\end{itemize}

\textbf{Monday, April 28}

\begin{itemize}
\item Work on Section 5.8
\end{itemize}

\textbf{Tuesday, April 29}

\begin{itemize}
\item Complete Section 5.8
\end{itemize}

\textbf{Wednesday, April 30}

\begin{itemize}
\item Review for Unit 6 Test
\end{itemize}

\textbf{Thursday, May 1}

\begin{itemize}
\item Take the Unit 6 Test
\end{itemize}

\textbf{Friday, May 2}

\begin{itemize}
\item Catch-up on MyMathLab (late penalty applies)
\end{itemize}

\textbf{Monday, May 5}

\begin{itemize}
\item \textbf{Submission Deadline for Worksheet \#8}
\item \textbf{Submission Deadline for Unit 6 Test}
\item \textbf{Submission Deadline for Unit 6 MyMathLab for full credit. Attempts for Unit 5 MyMathLab submitted after this date will recieve a 25\% point deduction.}
\end{itemize}

\textbf{Tuesday, May 6}

\begin{itemize}
\item Review for the Final Exam
\end{itemize}

\textbf{Wednesday, May 7}

\begin{itemize}
\item Review for the Final Exam
\end{itemize}

\textbf{Thursday, May 8}

\begin{itemize}
\item Review for the Final Exam
\end{itemize}

\textbf{Friday, May 9}

\begin{itemize}
\item In-person Final Exam
\item \textbf{Final submission deadline for all MyMathLab homework assignments. Any assignment not completed will be submitted as is, and any uncompleted assignments will be entered as a zero.}
\end{itemize}

\end{document}
