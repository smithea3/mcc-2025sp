% Created with jtex v.1.0.20
\documentclass[11pt]{article}
\usepackage{
    amsmath,
    fancyhdr,
    geometry,
    lastpage,
    xcolor
}

% You should have an imports section
%%%%%%%%%%%%%%%%%%%%%%%%%%%%%%%%%%%%%%%%%%%%%%%%%%
%%%%%%%%%%%%%%%%%%%%  imports  %%%%%%%%%%%%%%%%%%%
\usepackage{booktabs}
\usepackage{framed}
\usepackage{hyperref}
\usepackage{url}
%%%%%%%%%%%%%%%%%%%%%%%%%%%%%%%%%%%%%%%%%%%%%%%%%%


% Set document format
\setlength{\parindent}{0pt} % remove indent for paragraph
\setlength{\parskip}{6pt}   % change paragraph skip
\setcounter{secnumdepth}{0} % remove section numbering
\makeatletter
\renewcommand\section{\@startsection{section}{1}{0pt}%
  {-3.5ex \@plus -1ex \@minus -.2ex}% Space before
  {.2ex \@plus.2ex}% Space after
  {\normalfont\Large\bfseries}} % Style of the section
\makeatother

% Page setup
 \geometry{
    letterpaper,
    left=0.5in,
    right=0.5in,
    top=0.75in,
    bottom=1in
 }

\pagestyle{fancy}
\fancyhead{} % clear header
\fancyfoot{} % clear footer
\fancyhead[R]{MAT 272-SST1 (2025SP) Course Syllabus}
\fancyfoot[R]{Page \thepage\,of\,\pageref{LastPage}}
\fancyfoot[L]{Date Generated: \today}

% Define colors
\definecolor{mybgcolor}{RGB}{240, 240, 255} % Light pastel blue
\definecolor{mybordercolor}{RGB}{0, 51, 102} % Navy blue

\setlength{\FrameSep}{8pt} % Padding inside the frame
\setlength{\OuterFrameSep}{6pt} % Margin outside the frame
\renewenvironment{framed}[1][]{%
  \def\FrameCommand{%
    \hspace{1pt}%
    {\color{mybordercolor}\vrule width 2pt} % Border color and thickness
    \hspace{1pt}%
    \fboxsep=\FrameSep%
    \colorbox{mybgcolor}%
  }%
  \MakeFramed {\advance\hsize-\width \FrameRestore}%
}{%
  \endMakeFramed
}

% Define document parts/options
\title{MAT 272-SST1 (2025SP)}
% \abstract{}

\begin{document}

\begin{center}
    {\huge Mitchell Community College} \\[6pt]
    {\Large Calculus II} \\[6pt]
    {\Large MAT 272-SST1 (2025SP)}
\end{center}

\section{Course Information}

\textbf{Course Delivery:} Traditional Delevery Method\footnote{College curriculum or continuing education course in which 100\% of the instruction is delivered face to face with the instructor and student not  separated by distance. This is true even when some instructional activities are conducted using web‐based technology.}

\textbf{Class Location(s)/Day(s)/Time(s):}

\begin{itemize}
\item Monday and Wednesday: Bently Building\footnote{The Bently Building was previously known as the Vocational Building (VB).} 213 from 11 a.m. -- 11:50 a.m.
\item Tuesday and Thursday: Bently Building 213 from 11 a.m. -- 12:15 p.m.
\end{itemize}

\textbf{Institutional Credit Hours/Contact Hours:} 4 Credit, 3 Class, 2 Lab

\textbf{Prerequisites:} Please refer to the Mitchell Community College Catalog for more information.

\textbf{No Show Date (Census Date}): 08/28/2024

\section{Faculty Information}

\textbf{Instructor:} Ethan A. Smith, MA, MEd

\textbf{Email Address:} \href{mailto:esmith3@mitchellcc.edu}{esmith3@mitchellcc.edu}

\textbf{Response Time:} Mr. Smith will reply to emails and phone calls within 36 hours Monday through Friday. During the weekends, holidays, and institutional closing, response times may be increased due to lack of access to technology.

\textbf{Phone Number:} (704) 878-3270

\textbf{Office Location:} Bently Building, room 212A

\textbf{Available for Student Support:}

\begin{itemize}
\item Monday and Wednesday: 12 p.m. -- 1 p.m.
\item Tuesday and Thursday: 12:30 p.m. -- 2 p.m.
\end{itemize}

\textbf{College Reception Desks:}

\begin{itemize}
\item Statesville Campus: (704) 878-3200
\item Mooresville Campus: (704) 663-1923
\end{itemize}

\section{Catalog Description}

This course is designed to develop advanced topics of differential and integral calculus. Emphasis is placed on the applications of definite integrals, techniques of integration, indeterminate forms, improper integrals, infinite series, conic sections, parametric equations, polar coordinates, and differential equations. Upon completion, students should be able to select and use appropriate models and techniques for finding solutions to integral-related problems with and without technology. This course has been approved for transfer under the CAA as a general education course in Mathematics. This course has been approved for transfer under the ICAA as a general education course in Mathematics.

\section{Learning Management System (LMS)}

All curriculum courses have a space on Mitchell Community College's learning management system (LMS). The College's LMS is Moodle, which is hosted by Open LMS. Moodle is the college-wide adopted delivery portal for online course content. The College's Moodle site can be accessed from the \href{https://mitchellcc.edu/my-mitchell}{My Mitchell} page on the Mitchell Community College website or at \href{https://mycourses.mitchellcc.edu}{https://mycourses.mitchellcc.edu} by direct link.

\section{Learning Outcomes}

\begin{itemize}
\item \textbf{SLO1:} Select appropriate models and integration techniques to solve problems involving algebraic and transcendental functions; these problems will include but are not limited to applications involving volume, arc length, surface area, centroids, force and work.
\item \textbf{SLO2:} Apply appropriate models and integration techniques to solve problems involving algebraic and transcendental functions; these problems will include but are not limited to applications involving volume, arc length, surface area, centroids, force and work.
\item \textbf{SLO3:} Evaluate proper and improper integrals using various integration techniques.
\item \textbf{SLO4:} Analyze the convergence and divergence of infinite sequences and series.
\item \textbf{SLO5:} Find the Taylor and McLaurin representations for transcendental functions.
\item \textbf{SLO6:} Use differentiation and integration to analyze the graphs of polar form equations and parametric form equations.
\item \textbf{SLO7:} Solve separable and first-order linear differential equations.
\item \textbf{SLO8:} Analyze conic sections using calculus techniques.
\item \textbf{SLO9:} Graph conic sections using calculus techniques.
\item \textbf{SLO10:} Apply quantitative reasoning to solve problems.
\end{itemize}

\section{Recommendations for Success}

The definition of success can mean something different depending on the course. If you show up and engage with the material from the course, I will work with you to guide you to the goals and kinds of success you are wanting to achieve. It is important to remember the time~requirements~for~a~course for a course when making a plan for success.

Significant prior knowledge in algebra techniques and single variables calculus will greatly help your understanding and mastery of the material.

Below are some other recommendations for success in an online course from \href{https://online.osu.edu/resources/learn/5-online-learning-tips-student-success}{The Ohio State University} and the \href{https://www.colorado.edu/health/tips-succeeding-online-classes}{University of Colorado Boulder} that may be helpful advice for this course. However, this is \textbf{not} an online course.

\begin{enumerate}
\item Familiarize yourself with the technology.
\item Hold yourself accountable
\item Create a schedule and manage your time wisely.
\item Stay organized and be thorough.
\item Remain engaged throughout the whole course.
\item Take care of yourself.
\item Know where to turn for help.
\end{enumerate}

\begin{framed}
\textbf{Time Requirements of a College-level Course}\\
The amount of time a student is expected to dedicate to studying and completing assignments for a specific course, typically includes class time, homework, reading, and preparation for exams. As guideline, for traditional, 16-week courses, one credit hour equates to three hours per week completing course activities. Therefore, if you are taking a 16-week course worth four credit hours, you should expect to spend 12 hours per week working in the course. This time allotment increases as the number of weeks of the course decreases. Below is a table showing examples of various hours per week that is recommended to devote to the course based on the length of the course. You should be aware that these are \textit{suggested} times based on the credit hours and the length of the course. Various factors such as previous knowledge, difficulty of material, and extracurricular activites may alter these times. There is not a one-size-fit-all approach or rule for planning your success in a college-level course.

\bigskip\noindent
\begin{tabular}{p{\dimexpr 0.200\linewidth-2\tabcolsep}p{\dimexpr 0.200\linewidth-2\tabcolsep}p{\dimexpr 0.200\linewidth-2\tabcolsep}p{\dimexpr 0.200\linewidth-2\tabcolsep}p{\dimexpr 0.200\linewidth-2\tabcolsep}}
\toprule
Credit Hours & 4-week Course & 8-week Course & 12-week Course & 16-week Course \\
\hline
1 & 12 & 6 & 4.5 & 3 \\
2 & 24 & 12 & 9 & 6 \\
3 & 36 & 18 & 13.5 & 9 \\
4 & 48 & 24 & 18 & 12 \\
5 & 60 & 30 & 22.5 & 15 \\
\bottomrule
\end{tabular}

\bigskip
\end{framed}

\section{Required Textbook and Other Materials}

\subsection{Texbook}

Larson, R., B. Edwards. (2023). \textit{Calculus}. Boston: Cengage.

Note that a hard copy version of the textbook is optional, since it is available in electronic format with purchase of a WebAssign Access Code.

\subsection{Optional (Free) Resources}

\begin{itemize}
\item Strang, G., Herman, E., OpenStax College, \& Open Textbook Library. (2016). \textit{Calculus. Volume 2.} Openstax College. (Freely available to read and download at \href{https://openstax.org/details/books/calculus-volume-2}{openstax.org})


\item \href{https://www.youtube.com/watch?v=H9eCT6f\_Ftw\&list=PLDesaqWTN6EQ2J4vgsN1HyBeRADEh4Cw-\&index=1}{Professor Leonard's Calculus 2 Lecture Series on YouTube}
\end{itemize}

\subsection{Calculator}

\textbf{Physical Graphing Calculator:} TI-83/84 Plus family (or less) is recommended. Students may not use a calculator with a computer algebra system (CAS) built into the calculator.

\textbf{Desmos Graphing Calculator:} Students in this section of MAT-272 are allowed to use the \href{https://www.desmos.com/calculator}{Desmos Graphing Calculator} and the \href{http://www.desmos.com/3d}{Desmos 3D Grapching Calculator} during their assignments and during Calculator Active portions of their examinations.

\subsection{WebAssign}

You must purchase access to WebAssign. See the Assignment Description of this syllabus for more information about WebAssign.

\section{Teaching and Learning Strategies}

Mitchell uses Contextual Teaching and Learning (CTL) to enhance student learning through activities that connect academic concepts to relevant life experiences.

\section{Learning Environment}

This course is being offered on Mitchell Community College's Statesville campus in a traditional classroom. You are expected to attend class in-person at the scheduled class meeting time and location. Most class meetings will be lectures based on the material with opportunities for practice and skill reinforcement. Some class meetings will be devoted to the completion of assignments, quizzes, and examinations.

\section{Technology Requirements and Resources}

For information on Technology Requirements, visit the Mitchell Community College \href{https://www.mitchellcc.edu/college-credit-online-classes/\#online-learning}{Technology Requirements and Expectations} webpage (opens in a new window).

\section{Grading}

\subsection{Grading Scale}

\bigskip\noindent
\begin{tabular}{p{\dimexpr 0.500\linewidth-2\tabcolsep}p{\dimexpr 0.500\linewidth-2\tabcolsep}}
\toprule
Numeric Grade & Letter Grade \\
\hline
90-100\% & A \\
80-89\% & B \\
70-79\% & C \\
60-69\% & D \\
0-59\% & F \\
\bottomrule
\end{tabular}

\bigskip\subsection{Grading Breakdown}

Your grade is based on the following weighted categories and corresponding percentages.

\begin{itemize}
\item Homework (10\%)
\item Quizzes/Worksheets (15\%)
\item Chapter Tests (50\%)
\item Final Exam (25\%)
\end{itemize}

\subsection{Assignment Descriptions}

A brief description of assignments is provided below. Detailed information and requirements will be provided in class and/or online within the Learning Management System (LMS).

\textbf{Homework.} All homework assignments will be given through WebAssign. WebAssign is a fully customizable online instructional system empowering teachers to deploy assignments, assess individual student performance instantly, and achieve their teaching goals. Developed in 1997 and commercially available from 1998 under the guidance of founder and CEO Dr. John Risley, a physics education specialist. Students in this section of MAT-272 will access WebAssign through the specific links the LMS. Your grade on these assignments are automatically provided by the software and uploaded to the Gradebook of the LMS. Homework assignments must be completed by the due date, which should be no later than the date of the unit/chapter test to receive full credit. Students are able to receive 75\% credit on any Homework problems completed after the due date. The final submission date for Homework assignments is the Last Day of Classes.

\textbf{Quiz/Worksheets.} Students should expect quiz to be given during the scheduled class time and contain a combination of multiple choice and/or free response questions. Any calculator with a computer algebra system such as a TI-89 or TI-Inspire will not be allowed on quizzes. Some quizzes may be entirely calculator inactive or calculator active.

Students should expect worksheets questions to be any combination of multiple choice and/or free response. Worksheet assignments should be individual work. All written assignments should be neat and legible. Note that any calculator with a computer algebra system such as a TI-89 or TI-Inspire CAS will not be allowed on worksheets. Worksheets may have some flexibility in how they are submitted. Students may choose to submit their worksheet either in-person or online using the corresponding assignment link within the LMS by the posted deadline.

\begin{framed}
\textbf{Submitting Worksheets for Grading}\\
Some flexibility is granted on submitting worksheets for grading. Ssubmissions submitted through the link in the LMS will only be accepted by the posted deadline. Submissions sent via email will not be accepted or graded. If you experience issues submitting your worksheet assignments, please email your instructor as soon as possible. See Faculty~Information for your instructor's contact information.
\end{framed}

\textbf{Tests.} Students should expect tests to be administered in class. Some tests may have a take-home portion. If necessary, special circumstances may warrant a test to be given online through WebAssign. Tests will include multiple choice and/or free response questions. Tests will have two parts: calculator inactive and calculator active. In the event that a test has a take-home portion, the take-home portion counts as the calculator active portion. Note that any calculator with a computer algebra system such as a TI-89 or TI-Inspire will not be allowed on tests.

\textbf{Final Exam.} Students should expect the Final Exam to be administered in class during the last week of the course. The Final Exam will be cumulative and may include multiple choice and/or free response questions as well as calculator and/or non-calculator questions. Note that any calculator with a computer algebra system such as a TI-89 or TI-Inspire will not be allowed on the Final Exam.

\subsection{Missed Work}

Any assignments not completed by the due date is considered late. See Late~Work policy.

\subsection{Late Work}

There are no make-ups for assessments of learning --- i.e., worksheets, quizzes, or tests --- unless arrangements are made with the instructor prior to the due date. Prior arrangements, except for extreme, extenuating circumstance, is considered to be 24 hours before the due date of the assignment for which an extension is being requested. In most cases, request for extensions after a due date will not be granted.

\section{Faculty Feedback and Response Time}

\subsection{Grading and Feedback}

For most assessments, you can generally expect feedback within seven days. The only exception to this feedback/response time would be when an institutional holiday occurs.

Feedback on your WebAssign will be automatically generated by the software. If you are unclear why you missed a problem, you should reach out to your instructor.

Feedback on all handwritten assignments such as in-class worksheets, quizzes, and tests will be handwritten and returned either in class or electronically via the Learning Management System.

\section{Attendance Policy}

Mitchell Community College is an attendance taking institution. Instructors in all curriculum courses are required to report student attendance. Attendance begins on the first scheduled day of a course, even for students who register late. Mitchell Community College recognizes the connection between student attendance and student retention, achievement, and success. Students are expected to attend all class sessions, clinical experiences, and laboratory periods for which they are enrolled. Absence from any of these learning experiences, regardless of cause, reduces the opportunity for learning and may adversely affect a student's achievement.

Students are responsible for class attendance and for any class work missed during an absence. When a student fails to comply with the attendance policy of the class or fails to attend for two consecutive weeks (14 consecutive calendar days), the instructor should process an administrative withdrawal for the student resulting in a grade of W.

\subsection{Mathematics Department Attendance Policy for Traditional Courses}

If a student fails to attend class for 14 consecutive calendar days, then they will be withdrawn from the course.

\begin{framed}
\textbf{Defining Traditional Attendance}\\
A student is considered to be in compliance with the Mathematics Department Attendance Policy for Tradtional Courses if they physically come to the specificed time and location of the specific section in which they are registered.
\end{framed}

\section{Mitchell Community College Inclement Weather Policy}

In the event of adverse weather, the College will announce delays, cancellation of classes, or the closing of the college on local television and radio stations and on the \href{https://www.mitchellcc.edu}{College website}.

\section{No Show Date/Census Date Policy}

In order to remain enrolled in a course, a student must attend class on or before the class census date. If a student does not attend class by the census date, they will be reported as a \textbf{``no show'' (NS)} by the instructor and will be automatically withdrawn from the course.  To ensure students attend class and avoid being marked as a ``no show'', students need to:

\begin{itemize}
\item For traditional 100\% seated classes, a student must be physically present in class on or before the class census date.


\item For blended or hybrid classes, a student must either complete the Enrollment Verification Activity (EVA) in the LMS or physically be present in class on or before the class census date.


\item For 100\% online classes, a student must complete the EVA in the LMS on or before the class census date.
\end{itemize}

If a student does not meet the census date requirement, the student must be reported as a no-show for the class. Students reported as a ``no-show'' are withdrawn from the class.  The no show date and the census date are the same date for a course and can be found on the course syllabus as well as on the \href{https://mitchellcc.edu/office-student-records}{Office of Student Records webpage} (link opens in a new window). For blended, hybrid, and online courses this date is also noted in the EVA.

\section{Withdrawal Policy}

The last day a student can withdraw from a course or from all courses with a grade of ``W'' is at the 75 percent point of the course. The exact date is published on the \href{https://www.mitchellcc.edu/wp-content/uploads/2024/05/2024-2025-Academic-Calendar.pdf}{Academic Calendar} (link opens in new window). After the 75 percent point of the course, the student can no longer initiate a withdrawal and will receive the grade earned in the course at the end of the term. Students can locate withdrawal information by visiting the Office of Student Records webpage or the \href{https://www.mitchellcc.edu/schedule-adjustments-and-class-withdrawals/}{Schedule Adjustments and Class Withdrawals page}. If they need to withdraw from \textbf{all} of their classes they can click on the ``\href{https://mitchellcc.edu/withdrawal-procedure}{Withdrawal Procedure}'' menu link (links opens in a new window).

\section{Academic Dishonesty}

Mitchell Community College makes every reasonable effort to maintain integrity in all academic programs. To compromise integrity through acts of academic dishonesty jeopardizes the quality of instruction and the caliber of education we aim to provide our students.  Any form of academic dishonesty, by any student at the College, is unacceptable and will result in disciplinary action.

Definitions and Examples of Academic Dishonesty include, but may not be limited to:

\textbf{Cheating.} Intentionally and/or knowingly using unauthorized materials, information, or study aids in any academic exercise or matter.

\textbf{Plagiarism.} Intentionally and/or knowingly representing the words or ideas of another as one's own in any academic exercise or matter.

\textbf{Fabrication.} Intentionally and/or knowingly falsifying or inventing information or citations in an academic exercise or matter.

\textbf{Facilitating Academic Dishonesty.} Intentionally and/or knowingly helping or attempting to help another to commit an act of cheating, plagiarism, or fabrication.

\textbf{Self-Plagiarism.} The use of one's own previous work in another context without citing that it was used previously. The writer should let the reader know that this was not the first use of the material.

\pagebreak
\section{Tentative Schedule}

The following is a tentative schdule of the assigned readings and/or assessments for the course. All of these dates and learning activities are subject to change. The instructor will notify students of any change through email and the Announcements section of the Learning Management System (LMS) class site.

\textbf{Week of Monday, January 13}

\textbf{Week of Monday, January 20}

\begin{itemize}
\item The College is closed in observance of Martin Luther King Jr. Day. (No Class)
\end{itemize}

\textbf{Week of Monday, January 27}

\textbf{Week of Monday, February 03}

\textbf{Week of Monday, February 10}

\textbf{Week of Monday, February 17}

\textbf{Week of Monday, February 24}

\textbf{Week of Monday, March 03}

\textbf{Week of Monday, March 10}

\begin{itemize}
\item The week of March 10 is Spring Break. No classes are held; however, the College is open.
\end{itemize}

\textbf{Week of Monday, March 17}

\textbf{Week of Monday, March 24}

\textbf{Week of Monday, March 31}

\textbf{Week of Monday, April 07}

\textbf{Week of Monday, April 14}

\textbf{Week of Monday, April 21}

\textbf{Week of Monday, April 28}

\textbf{Week of Monday, May 05}

\textbf{Week of Monday, May 12}

\end{document}
