% Created with jtex v.1.0.20
\documentclass[11pt]{article}
\usepackage{
    amsmath,
    fancyhdr,
    geometry,
    lastpage,
    xcolor
}

% You should have an imports section
%%%%%%%%%%%%%%%%%%%%%%%%%%%%%%%%%%%%%%%%%%%%%%%%%%
%%%%%%%%%%%%%%%%%%%%  imports  %%%%%%%%%%%%%%%%%%%
\usepackage{booktabs}
\usepackage{framed}
\usepackage{hyperref}
\usepackage{url}
%%%%%%%%%%%%%%%%%%%%%%%%%%%%%%%%%%%%%%%%%%%%%%%%%%


% Set document format
\setlength{\parindent}{0pt} % remove indent for paragraph
\setlength{\parskip}{6pt}   % change paragraph skip
\setcounter{secnumdepth}{0} % remove section numbering
\makeatletter
\renewcommand\section{\@startsection{section}{1}{0pt}%
  {-3.5ex \@plus -1ex \@minus -.2ex}% Space before
  {.2ex \@plus.2ex}% Space after
  {\normalfont\Large\bfseries}} % Style of the section
\makeatother

% Page setup
 \geometry{
    letterpaper,
    left=0.5in,
    right=0.5in,
    top=0.75in,
    bottom=1in
 }

\pagestyle{fancy}
\fancyhead{} % clear header
\fancyfoot{} % clear footer
\fancyhead[R]{MAT-152-EPB1 (2025SP) Course Syllabus}
\fancyfoot[R]{Page \thepage\,of\,\pageref{LastPage}}
\fancyfoot[L]{Date Generated: \today}

% Define colors
\definecolor{mybgcolor}{RGB}{240, 240, 255} % Light pastel blue
\definecolor{mybordercolor}{RGB}{0, 51, 102} % Navy blue

\setlength{\FrameSep}{8pt} % Padding inside the frame
\setlength{\OuterFrameSep}{6pt} % Margin outside the frame
\renewenvironment{framed}[1][]{%
  \def\FrameCommand{%
    \hspace{1pt}%
    {\color{mybordercolor}\vrule width 2pt} % Border color and thickness
    \hspace{1pt}%
    \fboxsep=\FrameSep%
    \colorbox{mybgcolor}%
  }%
  \MakeFramed {\advance\hsize-\width \FrameRestore}%
}{%
  \endMakeFramed
}

% Define document parts/options
\title{MAT-152-EPB1 (2025SP)}
% \abstract{}

\begin{document}

\begin{center}
    {\huge Mitchell Community College} \\[6pt]
    {\Large Statistical Methods I} \\[6pt]
    {\Large MAT-152-EPB1 (2025SP)}
\end{center}

\section{Course Information}

\textbf{Course Delivery:} Blended Delevery Method\footnote{A Blended course is one in which $\leq$ 50\% of instruction is delivered online. Blended class sections primarily meet face-to-face on specified days and have a required online component, which requires students to have Internet access as part of the course.}

\textbf{Class Location(s)/Day(s)/Time(s):} Monday through Thursday at South Iredell High School, Classroom A-212 from 8:30 a.m. until 9:57 a.m.

\textbf{Institutional Credit Hours/Contact Hours:} 4 Credit, 3 Class, 2 Lab

\textbf{Prerequisites:} Please refer to the Mitchell Community College Catalog for more information.

\textbf{No Show Date (Census Date}): 01/21/2025

\section{Faculty Information}

\textbf{Instructor:} Ethan A. Smith, MA, MEd

\textbf{Email Address:} \href{mailto:esmith3@mitchellcc.edu}{esmith3@mitchellcc.edu}

\textbf{Response Time:} Mr. Smith will reply to emails and phone calls within 36 hours Monday through Friday. During the weekends, holidays, and institutional closing, response times may be increased due to lack of access to technology.

\textbf{Phone Number:} (704) 878-3270

\textbf{Office Location:} Bently Building, room 212A\footnote{The Bently Building was previously known as the Vocational Building (VB).}

\textbf{Available for Student Support:}

\begin{itemize}
\item Monday and Wednesday: 12 p.m. -- 1 p.m.
\item Tuesday and Thursday: 12:30 p.m. -- 2 p.m.
\end{itemize}

\textbf{College Reception Desks:}

\begin{itemize}
\item Statesville Campus: (704) 878-3200
\item Mooresville Campus: (704) 663-1923
\end{itemize}

\section{Catalog Description}

This course provides a project-based approach to introductory statistics with an emphasis on using real-world data and statistical literacy. Topics include descriptive statistics, correlation and regression, basic probability, discrete and continuous probability distributions, confidence intervals and hypothesis testing. Upon completion, students should be able to use appropriate technology to describe important characteristics of a data set, draw inferences about a population from sample data, and interpret and communicate results. This course has been approved for transfer under the CAA as a general education course in Mathematics (Quantitative). This course has been approved for transfer under the ICAA as a general education course in Mathematics (Quantitative).

\section{Learning Management System (LMS)}

All curriculum courses have a space on Mitchell Community College's learning management system (LMS). The College's LMS is Moodle, which is hosted by Open LMS. Moodle is the college-wide adopted delivery portal for online course content. The College's Moodle site can be accessed from the \href{https://mitchellcc.edu/my-mitchell}{My Mitchell} page on the Mitchell Community College website or at \href{https://mycourses.mitchellcc.edu}{https://mycourses.mitchellcc.edu} by direct link.

\section{Learning Outcomes}

\begin{itemize}
\item \textbf{SLO1:} Organize data.
\item \textbf{SLO2:} Construct appropriate graphs of data.
\item \textbf{SLO3:} Calculate descriptive statistics.
\item \textbf{SLO4:} Interpret descriptive statistics.
\item \textbf{SLO5:} Apply basic rules of probability.
\item \textbf{SLO6:} Identify appropriate probability distributions.
\item \textbf{SLO7:} Apply appropriate probability distributions.
\item \textbf{SLO8:} Perform regression analysis.
\item \textbf{SLO9:} Analyze sample data to draw inferences about a population parameter.
\item \textbf{SLO10:} Communicate results using a variety of methods.
\item \textbf{SLO11:} Apply quantitative reasoning to solve problems.
\end{itemize}

\section{Mitchell Essential Learning Outcome (MELO)}

This course may assess the Mitchell Essential Learning Outcome (MELO) of Quantitative Inquiry.

\section{Recommendations for Success}\label{recommendations-for-success}

The definition of success can mean something different depending on the course. If you show up and engage with the material from the course, I will work with you to guide you to the goals and kinds of success you work to achieve. It is important to remember the time~requirements~for~a~course for a course when making a plan for success.

Some prior knowledge of basic mathematics and number sense can help to increase your success.

Below are some other recommendations for success in an online course from \href{https://online.osu.edu/resources/learn/5-online-learning-tips-student-success}{The Ohio State University} and the \href{https://www.colorado.edu/health/tips-succeeding-online-classes}{University of Colorado Boulder}. Some of these recommendations are helpful and applicable to in-person courses as well.

\begin{enumerate}
\item Treat your online courses the same as in-person classes.
\item Familiarize yourself with the technology.
\item Hold yourself accountable.
\item Create a schedule and manage your time wisely.
\item Stay organized and be thorough.
\item Remain engaged throughout the whole course.
\item Take care of yourself.
\item Know where to turn for help.
\end{enumerate}

\begin{framed}
\textbf{Time Requirements of a College-level Course}\\
The amount of time a student is expected to dedicate to studying and completing assignments for a specific course, typically includes class time, homework, reading, and preparation for exams. As guideline, for traditional, 16-week courses, one credit hour equates to three hours per week completing course activities. Therefore, if you are taking a 16-week course worth four credit hours, you should expect to spend 12 hours per week working in the course. This time allotment increases as the number of weeks of the course decreases. Below is a table showing examples of various hours per week that is recommended to devote to the course based on the length of the course. You should be aware that these are \textit{suggested} times based on the credit hours and the length of the course. Various factors such as previous knowledge, difficulty of material, and extracurricular activites may alter these times. There is not a one-size-fit-all approach or rule for planning your success in a college-level course.

\bigskip\noindent
\begin{tabular}{p{\dimexpr 0.200\linewidth-2\tabcolsep}p{\dimexpr 0.200\linewidth-2\tabcolsep}p{\dimexpr 0.200\linewidth-2\tabcolsep}p{\dimexpr 0.200\linewidth-2\tabcolsep}p{\dimexpr 0.200\linewidth-2\tabcolsep}}
\toprule
Credit Hours & 4-week Course & 8-week Course & 12-week Course & 16-week Course \\
\hline
1 & 12 & 6 & 4.5 & 3 \\
2 & 24 & 12 & 9 & 6 \\
3 & 36 & 18 & 13.5 & 9 \\
4 & 48 & 24 & 18 & 12 \\
5 & 60 & 30 & 22.5 & 15 \\
\bottomrule
\end{tabular}

\bigskip
\end{framed}

\section{Required Textbook and Other Materials}

\subsection{Texbook}

Illowsky, B., \& Dean, S. L. (2023). \textit{Introductory statistics} (2e ed.). OpenStax~; Rice University.

A hard copy version of the textbook is optional, since it is freely available in electronic format at \href{https://openstax.org/details/books/introductory-statistics-2e?Book\%20details}{OpenStax} or directly at \href{https://openstax.org/details/books/introductory-statistics-2e?Book\%20details}{https://openstax.org/details/books/introductory-statistics-2e?Book details}.

\subsection{Optional (Free) Resources}

\begin{itemize}
\item \href{https://www.khanacademy.org/math/ap-statistics}{College/AP\texttrademark  Statistics Khan Academy} is good place to watch other short videos and practice problems.
\end{itemize}

\subsection{Calculator}

\textbf{Physical Graphing Calculator.} TI-83/84 Plus family (or less) is recommended, but not required. Students may not use a calculator with a computer algebra system (CAS) built into the calculator.

\textbf{Desmos Graphing Calculator.} Students in this section of MAT-152 are allowed to use the Desmos Graphing Calculator during their assignments and during the calculator active portiono of any exam. Students can access the Desmos Graphing Calculator at \href{https://www.desmos.com/calculator}{desmos.com/calculator}.

\textbf{Minitab.} Minitab is a powerful statistical software designed to simplify data analysis and visualization. It provides tools for statistical calculations, graphical representation of data, and process improvement, making it an essential resource for understanding and applying key concepts in statistics and data-driven decision-making.

Throughout this course, you will use Minitab to perform tasks such as analyzing datasets, creating graphs and charts, conducting hypothesis tests, and interpreting results. The software's intuitive interface and robust features will enable you to focus on understanding statistical methods and applying them to real-world problems. By using Minitab, you will gain hands-on experience with professional tools widely used in industries like business, engineering, healthcare, and research.

\subsection{MyOpenMath}

You will be required to access MyOpenMath to access your online homework problems. You will acess all of your MyOpenMath assignments directly through Open LMS. See the Assignment Description section of this syllabus for more information about MyOpenMath.

MyOpenMath takes student privacy and accessibility serious. You may read more about \href{https://asccc.org/sites/default/files/MyOpenMath\%20\_\%20WAMAP\%20\_\%20IMathAS\%20Accessibility.pdf}{MyOpenMath's Accessibility Compliance} and \href{https://www.myopenmath.com/info/policies/privacy.php}{Privacy Policy} on their website.

\section{Scanner or Scanner Software}

Some assignments may require you to have access to a scanner or a scanner application on your mobile device in order to upload certain assignments for this course to the LMS. There are several free scanner apps inside of the various ``stores'' for both Android and Apple devices. If you need assistance with scanning assignments onto your phone and uploading it to the respective place inside of Open LMS, you will need to reach out to your instructor for assistance.

\section{Teaching and Learning Strategies}

Mitchell uses Contextual Teaching and Learning (CTL) to enhance student learning through activities that connect academic concepts to relevant life experiences.

\section{Learning Environment}

This course is being offered on the campus of South Iredell High School in a traditional classroom. You are expected to attend class in-person at the scheduled class meeting time and location. Most class meetings will be lectures based on the material with opportunities for practice and skill reinforcement. Some class meetings will be devoted to the completion of lab assignments, quizzes, and examinations.

\section{Technology Requirements and Resources}

For information on Technology Requirements, visit the Mitchell Community College \href{https://www.mitchellcc.edu/college-credit-online-classes/\#online-learning}{Technology Requirements and Expectations} webpage (opens in a new window).

\section{Grading}

\subsection{Grading Scale}

\bigskip\noindent
\begin{tabular}{p{\dimexpr 0.500\linewidth-2\tabcolsep}p{\dimexpr 0.500\linewidth-2\tabcolsep}}
\toprule
Numeric Grade & Letter Grade \\
\hline
90-100\% & A \\
80-89\% & B \\
70-79\% & C \\
60-69\% & D \\
0-59\% & F \\
\bottomrule
\end{tabular}

\bigskip\subsection{Grading Breakdown}

Your grade is based on the following weighted categories and corresponding percentages.

\begin{itemize}
\item Homework (15\%)
\item Quizzes/Labs (25\%)
\item Exams (40\%)
\item Midterm (10\%)
\item Final Exam (10\%)
\end{itemize}

\subsection{Assignment Descriptions}

A brief description of assignments is provided below. Detailed information and requirements will be provided in class and/or online within the LMS. Each of the following assignments are expected to be submitted to their respective location within the LMS. No submission through email will be accepted.

\textbf{Homework.} Homework assignments will be given through MyOpenMath. MyOpenMath is an online learning platform that combines trusted author content with digital tools specifically designed to supplement mathematics textbooks and instruction. The questions in MyOpenMath are randomized for each student; therefore, it is highly unlikely that any two students will have the same exact MyOpenMath problems.

MyOpenMath assignments must be completed by the due date listed in MyOpenMath (which should be no later than the date of the unit test) in order to receive full credit. Students are able to receive 75 percent credit on any MyOpenMath problems completed after the due date. The final submission date to receive partial credit for MyOpenMath is the Last Day of Classes.

\textbf{Quizzes/Labs.} Students should expect quiz to consist of any combination of multiple choice and/or free response questions. Quizzes will be given in class and may be contain calculator inactive and calculator active portions.

Lab assignments are to assess your ability to synthesize and apply the information learned in class. Unless otherwise stated by the instructor, lab assignments should be individual work. All work should be completed in a digital format that can be uploaded to Open LMS for grading. Some labs will require you to utlize Minitab for statistical analysis.

\begin{framed}
\textbf{Submitting Labs for Grading}\\
Only submissions submitted through the link in the LMS will be graded. Submissions sent via email will not be accepted or graded. If you experience issues submitting your lab assignments, please email your instructor as soon as possible. See Faculty~Information for your instructor's contact information.
\end{framed}

\textbf{Unit Exams.} Unit Exams will include multiple choice and/or free response questions. Note that any calculator with a computer algebra system such as a TI-89 or TI-Inspire are not be allowed on exams.

\textbf{Midterm.} The Midterm Exam will be given at the half-way point of the course and contain a calculator inactive and a calculator active portion.

\textbf{Final Exam.} The Final Exam will be given at the conclusion of the course and contain a calculator inactive and a calculator active portion. The material covered on the Final Exam will consist of the units covered after the Midterm Exam.

\subsection{Missed Work}

Any assignments not completed by the due date is considered late. See Late~Work policy.

\subsection{Late Work}\label{late-work}

There are no make-ups for missed Labs, Quizzes, the Midterm Exam or the Final Exam unless arrangements are made with the instructor prior to the due date. Prior arrangements, except for extreme, extenuating circumstance, is considered to be 24 hours before the due date of the assignment for which an extension is being requested. In most cases, request for extensions after a due date will not be granted.

\section{Faculty Feedback and Response Time}

\subsection{Grading and Feedback}

For most assessments, you can generally expect feedback within seven days. The only exception to this feedback/response time would be when an institutional holiday occurs.

Feedback on your MyOpenMath will be automatically generated by the software. If you are unclear why you missed a problem, you should reach out to your instructor.

Feedback on all handwritten assignments such as quizzes and exams will be handwritten and returned either in class or electronically via the Learning Management System, depending on how the assignment was submitted.

\section{Attendance Policy}

Mitchell Community College is an attendance taking institution. Instructors in all curriculum courses are required to report student attendance. Attendance begins on the first scheduled day of a course, even for students who register late. Mitchell Community College recognizes the connection between student attendance and student retention, achievement, and success. Students are expected to attend all class sessions, clinical experiences, and laboratory periods for which they are enrolled. Absence from any of these learning experiences, regardless of cause, reduces the opportunity for learning and may adversely affect a student's achievement.

Students are responsible for class attendance and for any class work missed during an absence. When a student fails to comply with the attendance policy of the class or fails to attend for two consecutive weeks (14 consecutive calendar days), the instructor should process an administrative withdrawal for the student resulting in a grade of W.

\subsection{Mathematics Department Attendance Policy for Blended Courses}

Attendance for this blended course is in two parts: online and seated.  You must be compliant with both components in order to remain in the class.

\textbf{Online component.} If you fail to be active in the course for 14 consecutive calendar days in the online component, then you will be withdrawn from the course.

\textbf{Seated component.} If you fail to attend class for two consecutive weeks, then you will be withdrawn from the course.

\begin{framed}
\textbf{Defining Active Participation in a Blended Course}\\
A blended courses consists of two modes of instruction: seated and online.

To be counted present for the seated component, you must phyiscally attend the scheduled course meeting.

For the online component, active participation can be derived from several components such as homework completion or lab submission. The following provides some guidance on active participation in the online component of the course.

Activities that \textbf{do} count towards active participation in the online component include

\begin{itemize}
\item Making progress in a MyOpenMath assignment for that week.
\item Submitting a lab by the posted due date.
\end{itemize}

Activities that \textbf{do not} count towards active participation include

\begin{itemize}
\item Logging into the LMS without doing any of the above activities.
\item Watching a lecture video.
\item Clicking on an assignment, but not submitting any work.
\end{itemize}
\end{framed}

\section{Mitchell Community College Inclement Weather Policy}

In the event of adverse weather, the College will announce delays, cancellation of classes, or the closing of the college on local television and radio stations and on the \href{https://www.mitchellcc.edu}{College website}.

\section{No Show Date/Census Date Policy}

In order to remain enrolled in a course, a student must attend class on or before the class census date. If a student does not attend class by the census date, they will be reported as a \textbf{``no show'' (NS)} by the instructor and will be automatically withdrawn from the course.  To ensure students attend class and avoid being marked as a ``no show'', students need to:

\begin{itemize}
\item For traditional 100\% seated classes, a student must be physically present in class on or before the class census date.


\item For blended or hybrid classes, a student must either complete the Enrollment Verification Activity (EVA) in the LMS or physically be present in class on or before the class census date.


\item For 100\% online classes, a student must complete the EVA in the LMS on or before the class census date.
\end{itemize}

If a student does not meet the census date requirement, the student must be reported as a no-show for the class. Students reported as a ``no-show'' are withdrawn from the class.  The no show date and the census date are the same date for a course and can be found on the course syllabus as well as on the \href{https://mitchellcc.edu/office-student-records}{Office of Student Records webpage} (link opens in a new window). For blended, hybrid, and online courses this date is also noted in the EVA.

\section{Withdrawal Policy}

The last day a student can withdraw from a course or from all courses with a grade of ``W'' is at the 75 percent point of the course. The exact date is published on the \href{https://www.mitchellcc.edu/wp-content/uploads/2024/05/2024-2025-Academic-Calendar.pdf}{Academic Calendar} (link opens in new window). After the 75 percent point of the course, the student can no longer initiate a withdrawal and will receive the grade earned in the course at the end of the term. Students can locate withdrawal information by visiting the Office of Student Records webpage or the \href{https://www.mitchellcc.edu/schedule-adjustments-and-class-withdrawals/}{Schedule Adjustments and Class Withdrawals page}. If they need to withdraw from \textbf{all} of their classes they can click on the ``\href{https://mitchellcc.edu/withdrawal-procedure}{Withdrawal Procedure}'' menu link (links opens in a new window).

\section{Academic Dishonesty}

Mitchell Community College makes every reasonable effort to maintain integrity in all academic programs. To compromise integrity through acts of academic dishonesty jeopardizes the quality of instruction and the caliber of education we aim to provide our students.  Any form of academic dishonesty, by any student at the College, is unacceptable and will result in disciplinary action.

Definitions and Examples of Academic Dishonesty include, but may not be limited to:

\textbf{Cheating.} Intentionally and/or knowingly using unauthorized materials, information, or study aids in any academic exercise or matter.

\textbf{Plagiarism.} Intentionally and/or knowingly representing the words or ideas of another as one's own in any academic exercise or matter.

\textbf{Fabrication.} Intentionally and/or knowingly falsifying or inventing information or citations in an academic exercise or matter.

\textbf{Facilitating Academic Dishonesty.} Intentionally and/or knowingly helping or attempting to help another to commit an act of cheating, plagiarism, or fabrication.

\textbf{Self-Plagiarism.} The use of one's own previous work in another context without citing that it was used previously. The writer should let the reader know that this was not the first use of the material.

\pagebreak
\section{Tentative Schedule}

The following is a tentative schdule of the assigned readings and/or assessments for the course. All of these dates and learning activities are subject to change. The instructor will notify students of any change through email and the Announcements section of the Learning Management System (LMS) class site.

\textbf{Monday, January 13}

\begin{itemize}
\item First Day of Class
\item Syllabus Discussion
\item Section 1.1: Definitions of Statistics, Probability, and Key Terms
\end{itemize}

\textbf{Tuesday, January 14}

\begin{itemize}
\item Section 1.2: Data, Sampling, and Variation in Data and Sampling
\end{itemize}

\textbf{Wednesday, January 15}

\begin{itemize}
\item Section 1.3: Frequency, Frequency Tables, and Levels of Measurement
\end{itemize}

\textbf{Thursday, January 16}

\begin{itemize}
\item Section 1.4: Experimental Design and Ethics
\item Application Lab: Data Collection | Deadline: Tuesday, January 21 by 8 a.m.
\end{itemize}

\textbf{Monday, January 20}

\begin{itemize}
\item \textbf{Martin Luther King Jr. Day | College Closed | No Class}
\end{itemize}

\textbf{Tuesday, January 21}

\begin{itemize}
\item Section 2.1: Stem-and-Leaf Graphs (Stemplots), Line Graphs, and Bar Graphs
\item Section 2.2: Histograms, Frequency Polygons, and Time Series Graphs
\item Complete Chapter 1 Quiz (in MyOpenMath)
\end{itemize}

\textbf{Wednesday, January 22}

\begin{itemize}
\item Section 2.3: Measures of the Location of the Data
\item Section 2.4: Box Plots
\end{itemize}

\textbf{Thursday, January 23}

\begin{itemize}
\item Section 2.5: Measures of the Center of the Data
\item Section 2.6: Skewness and the Mean, Median, and Mode
\end{itemize}

\textbf{Monday, January 27}

\begin{itemize}
\item Section 2.7: Measures of the Spread of the Data
\item Section 2.8: Descriptive Statistics
\item Application Lab: | Deadline:
\item Complete Chapter 2 Quiz (in MyOpenMath)
\end{itemize}

\textbf{Tuesday, January 28}

\begin{itemize}
\item Finish Unit 1 MyOpenMath assignments
\item Review for Unit 1 Exam
\end{itemize}

\textbf{Wednesday, January 29}

\begin{itemize}
\item Complete Unit 1 Exam
\item All MyOpenMath homework assignments for Unit 1 are due by 11:59 p.m. to avoid the late penalty.
\end{itemize}

\textbf{Thursday, January 30}

\begin{itemize}
\item Section 3.1: Probability Terminology
\end{itemize}

\textbf{Monday, February 3}

\begin{itemize}
\item Section 3.2: Independent and Mutually Exclusive Events
\item Section 3.3: Two Basic Rules of Probability
\end{itemize}

\textbf{Tuesday, February 4}

\begin{itemize}
\item Section 3.4: Contingency Tables
\end{itemize}

\textbf{Wednesday, February 5}

\begin{itemize}
\item Section 3.5 Tree and Venn Diagrams
\item Application Lab: Probability
\end{itemize}

\textbf{Thursday, February 6}

\begin{itemize}
\item Section 4.1: Probability Distribution Function (PDF) for a Discrete Random Variable
\item Complete Chapter 3 Quiz (in MyOpenMath)
\end{itemize}

\textbf{Monday, February 10}

\begin{itemize}
\item Section 4.2: Mean or Expected Value and Standard Deviation
\end{itemize}

\textbf{Tuesday, February 11}

\begin{itemize}
\item Section 4.3: Binomial Distribution
\end{itemize}

\textbf{Wednesday, February 12}

\begin{itemize}
\item Finish Unit 2 MyOpenMath assignments
\item Review for Unit 2 Exam
\item Complete Chapter 4 Quiz (in MyOpenMath)
\end{itemize}

\textbf{Thursday, February 13}

\begin{itemize}
\item Complete Unit 2 Exam
\item All MyOpenMath homework assignments for Unit 2 are due by 8 a.m. to avoid the late penalty.
\end{itemize}

\textbf{Monday, February 17}

\begin{itemize}
\item Section 5.1: Continuous Probability Functions
\end{itemize}

\textbf{Tuesday, February 18}

\begin{itemize}
\item Section 5.2: The Uniform Distribution
\end{itemize}

\textbf{Wednesday, February 19}

\begin{itemize}
\item Section 6.1: The Standard Normal Distribution
\item Complete Chapter 5 Quiz (in MyOpenMath)
\end{itemize}

\textbf{Thursday, February 20}

\begin{itemize}
\item Section 6.2: Using the Normal Distribution
\item Application Lab: | Deadline:
\end{itemize}

\textbf{Monday, February 24}

\begin{itemize}
\item Section 7.1: The Central Limit Theorem for Sample Means (Averages)
\item Complete Chapter 6 Quiz (in MyOpenMath)
\end{itemize}

\textbf{Tuesday, February 25}

\begin{itemize}
\item Section 7.2: The Central Limit Theorem for Sums
\end{itemize}

\textbf{Wednesday, February 26}

\begin{itemize}
\item Section 7.3: Using the Central Limit Theorem
\end{itemize}

\textbf{Thursday, February 27}

\begin{itemize}
\item Finish Unit 3 MyOpenMath assignments
\item Review for Unit 3 Exam
\item Complete Chapter 7 Quiz (in MyOpenMath)
\end{itemize}

\textbf{Monday, March 3}

\begin{itemize}
\item Complete Unit 3 Exam
\item All MyOpenMath homework assignments for Unit 3 are due by 8 a.m. to avoid the late penalty.
\end{itemize}

\textbf{Tuesday, March 4}

\begin{itemize}
\item Section 8.1: A Single Population Mean using the Normal Distribution
\end{itemize}

\textbf{Wednesday, March 5}

\begin{itemize}
\item Section 8.2: A Single Population Mean using the Student $t$ Distribution
\end{itemize}

\textbf{Thursday, March 6}

\begin{itemize}
\item Midterm Exam
\end{itemize}

\textbf{Monday, March 10}

\begin{itemize}
\item \textbf{Spring Break | No Class | College Open}
\end{itemize}

\textbf{Tuesday, March 11}

\begin{itemize}
\item \textbf{Spring Break | No Class | College Open}
\end{itemize}

\textbf{Wednesday, March 12}

\begin{itemize}
\item \textbf{Spring Break | No Class | College Open}
\end{itemize}

\textbf{Thursday, March 13}

\begin{itemize}
\item \textbf{Spring Break | No Class | College Open}
\end{itemize}

\textbf{Monday, March 17}

\begin{itemize}
\item Section 8.3: A Population Proportion
\end{itemize}

\textbf{Tuesday, March 18}

\begin{itemize}
\item Section 9.1: Null and Alternative Hypotheses
\item Section 9.2: Outcomes and the Type I and Type II Errors
\item Complete Chapter 8 Quiz (in MyOpenMath)
\end{itemize}

\textbf{Wednesday, March 19}

\begin{itemize}
\item Section 9.3: Probability Distribution Needed for Hypothesis Testing
\end{itemize}

\textbf{Thursday, March 20}

\begin{itemize}
\item Section 9.4: Rare Events, the Sample, Decision and Conclusion
\item Section 9.5: Additional Information and Full Hypothesis Test Examples
\item Application Lab: Hypothesis Testing| Deadline:
\end{itemize}

\textbf{Monday, March 24}

\begin{itemize}
\item Section 11.1: Facts About the Chi-Square Distribution
\item Complete Chapter 9 Quiz (in MyOpenMath)
\end{itemize}

\textbf{Tuesday, March 25}

\begin{itemize}
\item Section 11.2: Goodness-of-Fit Test
\item Review for Unit 4 Test
\item Complete Chapter 9 and Chapter 11 Quiz (in MyOpenMath)
\end{itemize}

\textbf{Wednesday, March 26}

\begin{itemize}
\item \textbf{South Iredell High School Early Release | No Class | College Open}
\end{itemize}

\textbf{Thursday, March 27}

\begin{itemize}
\item Complete Unit 4 Test
\item All MyOpenMath homework assignments for Unit 4 are due by 8 a.m. to avoid the late penalty.
\end{itemize}

\textbf{Monday, March 31}

\begin{itemize}
\item Section 12.1: Linear Equations
\item Section 12.2: Scatter Plots
\end{itemize}

\textbf{Tuesday, April 1}

\begin{itemize}
\item Section 12.3: The Regression Equation
\end{itemize}

\textbf{Wednesday, April 2}

\begin{itemize}
\item Section 12.4: Testing the Significance of the Correlation Coefficient
\end{itemize}

\textbf{Thursday, April 3}

\begin{itemize}
\item Section 12.5: Prediction
\end{itemize}

\textbf{Monday, April 7}

\begin{itemize}
\item Section 12.6: Outliers
\item Complete Application Lab: Regression | Deadline:
\end{itemize}

\textbf{Tuesday, April 8}

\begin{itemize}
\item Review for Unit 5 Exam
\item Complete Chapter 12 Quiz (in MyOpenMath)
\end{itemize}

\textbf{Wednesday, April 9}

\begin{itemize}
\item Complete Unit 5 Exam
\item All MyOpenMath homework assignments for Unit 4 are due by 8 a.m. to avoid the late penalty.
\end{itemize}

\textbf{Thursday, April 10}

\begin{itemize}
\item Review for Final Exam
\end{itemize}

\textbf{Monday, April 14}

\begin{itemize}
\item Last Day of Class
\item Complete the Final Exam
\end{itemize}

\end{document}
